
\begin{minipage}{.39\textwidth}
    через <<л>> (<<ложь>>), <<f>> (<<false>>) или <<0>>. Каждое высказывание, как и высказывания, из которых оно состоит, (и от которых, в зависимости от способа их соединения, зависит его значение), может принимать два различных значения, вовсе не обязательно называемые <<истиной>> или <<ложью>> и ассоциируемые с этими понятиями. Таким образом, исчисление высказываний можно понимать как \emph{<<алгебру логики>>}~---исследование функций, принимающих, так же как и их аргументы, два различных значения. Эти значения можно (но, повторяем, вовсе не обязательно) называть истиной или ложью. Если мы не только применяем эти наименования, но и интересуемся их связью с <<обычными>> понятиями истины и лжи (и пытаемся уточнить их), то мы занимаемся \emph{<<семантикой>>}.

    \hspace{0.5cm}В противном случае мы, оставаясь в рамках чистого \emph{синтаксиса}, можем вообще спокойно забыть о происхождении нашей <<логической>> терминологии. Такое <<чистое>> исчисление высказываний есть попросту раздел \emph{комбинаторики}, и <<логические задачи>>, рассматриваемые в нём, ничем, в принципе не отличаются от обычных комбинаторных задач.

    \hspace{0.5cm}Итак, кроме самих по себе высказываний, исчисление высказываний изучает различные функции между ними~---различные способы образования <<сложных>> высказываний из <<простых>>. В рамках семантики эти функции очень напоминают обычные союзы русского (или любого другого) языка, с помощью которых сложные предложения строятся из простых. Но эту связь с обычным языком (и с <<обычной логикой>>) мы отложим до следующего разговора. Поэтому мы закончим нашу статью тем, чем обычно подобные статьи начинаются: определением нескольких <<основных>> таких функций, или, как их называют, \emph{логических операций}.

    \hspace{0.5cm}Поскольку речь идёт о функциях, принимающих, как их аргументы, конечное число значений (а именно
\end{minipage}
\hspace{0.5cm}
\begin{minipage}{.5\textwidth}
    два), мы сможем задать интересующие нас функции явным указанием на то, какие именно значения они принимают при всевозможных распределениях значений элементов (<<простых составляющих высказываний>>). Такие задания-определения удобно представлять в виде так называемых \emph{истинностных таблиц}, на <<входах>> которых указаны значения исходных высказываний, а на <<выходах>> (в клетках самой таблицы) ~--- значения результирующего высказывания.

    \hspace{0.5cm}Вот эти логические операции.

    \hspace{0.5cm}1) \emph{Отрицание} истринного высказывания ложно, а отрицание ложного высказывания истинно. Это, обозначаемая символом <<$\neg$>> операция, соответствующая частице <<не>> в русском языке, задаётся следующей истинностной таблицей:

    \begin{center}
        \begin{tabular}{@{\hspace{.23\textwidth}}c@{\hspace{.23\textwidth}}|@{\hspace{.23\textwidth}}c@{\hspace{.23\textwidth}}}
             $A$ & $\neg A$\\
             \hline
             &\\
             и & л\\
             л & и\\
        \end{tabular}
    \end{center}

    \hspace{0.5cm}2) \emph{Конъюнкция} двух истинных высказываний (соответствующая союзу <<и>> между ними, обозначение~--- $\bigwedge$) истинна, если же хотя бы одно из них ложно~--- ложна:

    \begin{center}
        \begin{tabular}{@{\hspace{.14\textwidth}}c@{\hspace{.14\textwidth}}|@{\hspace{.14\textwidth}}c@{\hspace{.14\textwidth}}|@{\hspace{.14\textwidth}}c@{\hspace{.14\textwidth}}}
             $A$ & $B$ & $A\bigwedge B$\\
             \hline
             &&\\
             и & и & и\\
             и & л & л\\
             л & и & л\\
             л & л & л
        \end{tabular}
    \end{center}

    \hspace{0.5cm}3) \emph{Дизьюнкция} двух высказываний (обозначение~-- $\bigvee$, читается как <<или>>) истинна, если истинно хотя бы одно из них, и ложна, если оба они ложны:
    \begin{center}
        \begin{tabular}{@{\hspace{.14\textwidth}}c@{\hspace{.14\textwidth}}|@{\hspace{.14\textwidth}}c@{\hspace{.14\textwidth}}|@{\hspace{.14\textwidth}}c@{\hspace{.14\textwidth}}}
             $A$ & $B$ & $A\bigvee B$\\
             \hline
             &&\\
             и & и & и\\
             и & л & и\\
             л & и & и\\
             л & л & л
        \end{tabular}
    \end{center}
    \centering{\textit{(Окончание см с 35)}}
\end{minipage}
