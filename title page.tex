\thispagestyle{empty}
\begin{center}
    Федеральное государственное автономное образовательное учреждение\\ 
    высшего образования\\
    «Национальный исследовательский университет ИТМО»\\
    \textit{Факультет Программной Инженерии и Компьютерной Техники}\\
\end{center}
\vspace{2cm}
\begin{center}
    \large
    \textbf{Лабораторная работа № 6}\\
    по дисциплине информатика\\
    Работа с системой компьютерной вёрстки \LaTeX \\
    Вариант № 51
\end{center}
\vspace{7cm}
\begin{flushright}
    Выполнил:\\
    cтудент  группы P3115\\
    Ахунов А.А.\\
    Преподаватель: \\
    Малышева Т.А.\\
\end{flushright}
\vspace{6cm}
\begin{center}
    г. Санкт-Петербург\\
    2022г.
\end{center}
\newpage

\tableofcontents

\newpage

\section{Задание}
Год выпуска: 1975\\
Выпуск: 1\\
Страницы: 29, 45\\

\href{https://kvant.ras.ru/pages.html#1975&1&29}{29 страница}

\href{https://kvant.ras.ru/pages.html#1975&1&45}{45 страница}

\newpage

\section{Выполнение работы}
\url{https://github.com/AmaHah/Informatics-Lab6}
\newpage

\section{Вывод}
В ходе работы я научился работать в системе \LaTeX и узнал интересные факты из журнала Квант.
